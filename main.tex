\documentclass[10pt,a4paper]{article}
\usepackage[utf8]{inputenc}
\usepackage[T1]{fontenc}
\usepackage{geometry}
\geometry{a4paper, margin=2.5cm}
\usepackage{graphicx}
\usepackage{amsmath,amssymb,amsthm}
\usepackage{hyperref}
\usepackage{setspace}
\usepackage{titlesec}
\usepackage{natbib}
\usepackage{xcolor}
\usepackage{fancyhdr}

% Springer Nature style colors
\definecolor{SNBlue}{RGB}{0,102,153}

% Line spacing
\onehalfspacing

% Title formatting - unnumbered sections
\titleformat{\section}{\normalfont\Large\bfseries\color{SNBlue}}{}{0em}{}
\titleformat{\subsection}{\normalfont\large\bfseries}{}{0em}{}
\titleformat{\subsubsection}{\normalfont\normalsize\bfseries}{}{0em}{}

% Custom commands for Springer Nature style
\newcommand{\fnm}[1]{\textbf{#1}}
\newcommand{\sur}[1]{\textbf{#1}}
\newcommand{\email}[1]{\texttt{#1}}

% Author and affiliation formatting
\newcounter{affil}
\newcommand{\affil}[2][]{%
  \ifx\relax#1\relax
    \textsuperscript{#1}#2%
  \else
    \textsuperscript{#1}#2%
  \fi
}

% Abstract environment
\renewenvironment{abstract}{%
  \small
  \begin{center}%
    {\bfseries \abstractname\vspace{-.5em}\vspace{0pt}}%
  \end{center}%
  \quotation
}{%
  \endquotation
}

% Keywords command
\newcommand{\keywords}[1]{%
  \vspace{0.5cm}
  \noindent\textbf{Keywords:} #1
  \vspace{0.5cm}
}

% Remove page numbers for draft
\pagestyle{plain}
\pagenumbering{arabic}

\begin{document}

% Title
\begin{center}
    {\LARGE\bfseries\color{SNBlue} Towards a Community of Practice: Strengthening Infectious Disease Modelling Capacity in China}
\end{center}

\vspace{1cm}

% Authors - commented out for now
% \begin{center}
%     \textbf{[Author Name]\textsuperscript{1}$^{*}$}\quad\textbf{[Co-author Name]\textsuperscript{2}}
% \end{center}
%
% \vspace{0.3cm}
%
% % Affiliations
% \begin{center}
%     \textsuperscript{1}[Your Affiliation] \\
%     \textsuperscript{2}[Co-author Affiliation] \\
%     \vspace{0.2cm}
%     \textsuperscript{*}\textbf{Corresponding Author:} [Author Name] \\
%     Email: \texttt{[your.email@example.com]}
% \end{center}

\vspace{0.5cm}

\section{Overview}\label{sec1}
\subsection*{Key Points}

\begin{enumerate}
    \item{Learning from experiences during pandemic response across LMIC contexts.}
    \item{Modelling capacity for infectious disease decision making in LMICs remains fragmented compared with high-income settings, though targeted investments are growing.}
    \item{Briefly introduce the community workshop---its partners, participants, and aims as an LMIC-focused learning exchange.}
\end{enumerate}

\subsection*{Notes}
\begin{itemize}
    \item{[Add contextual notes or evidence references.]}
\end{itemize}
\subsection*{TODO}
\begin{itemize}
    \item{Please refer to the two papers by Katherine Sherratt on assessing the IDM activity in the UK: \cite{Sherratt2024, Sherratt2025}.}
    \item{Mention the model use in policy lancet commission written by Mark et al.: \cite{Leung2025}.}
    \item{Mention the situation analysis of infectious disease modelling in Thailand: \cite{Sittimart2025}.}
\end{itemize}

\section{Aligning Modelling Complexity With Real Decision Needs}\label{sec2}
\subsection*{Key Points}

\begin{enumerate}
    \item{There needs to be a balance between what is of research interest and what is of decision/ policy interest. For modellers specifically, this is a balance between complexity and operationality.}
    \item{In a decision/ policy support context, key research questions need to be identified together, between modellers and decision makers. This process needs to be iterative and collaborative.}
\end{enumerate}

\subsection*{Notes}
\begin{itemize}
  \item{Not all IDMs are intended to support decision-making. But those that do often are considered most impactful.}
  \item{Modellers sometimes chase after complexity, but complex models are not necessarily the best fit to support decision-making.}
  \item{How will decision makers understand what evidence they need?}  
  \item{Will decision makers use the evidence from a model if they do not necessarily understand it?}
  \item{Decision-informing models need to be dynamic - they need to be able to incorporate new data and knowledge as they become available.}
\end{itemize}
\subsection*{TODO}
\begin{itemize}
    \item{JK should go back to the database and see if there are some interesting quotes that you can pull to support these points.}
\end{itemize}

\section{Data}\label{sec3}
\subsection*{Key Points}
\begin{enumerate}
    \item{Data quality}
    \item{Data accessibility}
    \item{Data representativeness}
\end{enumerate}
\subsection*{Notes}
\begin{itemize}
    \item{[Summarise supporting evidence or workshop anecdotes.]}
\end{itemize}
\subsection*{TODO}
\begin{itemize}
    \item{[List analyses, figures, or interviews to include.]}
\end{itemize}

\section{Interdisciplinary Collaboration}\label{sec4}
\subsection*{Key Points}
\begin{enumerate}
    \item{IDM development is intrinsically interdisciplinary.}
    \item{This is where we can talk about the experience from the Canada experience - the language people are using are different.}
    \item{We already discussed the problems and challenges with data - adding the interdisciplinary layer onto this makes it an even greater challenge.}
\end{enumerate}
\subsection*{Notes}
\begin{itemize}
    \item{[Capture stakeholder perspectives or quotes to highlight tensions.]}
\end{itemize}
\subsection*{TODO}
\begin{itemize}
    \item{[Clarify which examples (e.g., Canada case) will be expanded.]}
\end{itemize}

% Jennifer will revisit the script for this section.
\section{AI in Modelling}\label{sec5}
\subsection*{Key Points}
\begin{enumerate}
    \item{AI is obviously a powerful tool for IDM development but it is not ready to replace humans yet. 
    It remains limited in producing transparent, reliable, and diversified models.}
    \item{Integrating AI into IDM developments requires computation power and performance tests.}
    \item{Ethical considerations. We need ``responsible use'' of AI in modelling - but what is being responsible mean?}
    \item{Working with AI now but also protects privacy and data security.}
    %TODO: @jinke what does it mean when they say they require "training tests"?
    \item{AI best practices as example? But who should be publishing these?}
\end{enumerate}
\subsection*{Notes}
\begin{itemize}
    \item{[Clarify open questions such as ``training tests'' requirement.]}
\end{itemize}
\subsection*{TODO}
\begin{itemize}
    \item{[Identify experts to interview on AI governance and infrastructure.]}
\end{itemize}

% Jennifer will revisit the script for this section.
\section{Building a Coherent and Continuous Training Ecosystem for IDM Capacity}\label{sec6}
\subsection*{Key Points}
\begin{enumerate}
    \item{AI-assisted learning may need to be carefully thought out. Good to have different standards of practice.}
    \item{Capcity building and strengthening is segmented.}
    \item{Capacity building must go beyond technical skills to include leadership, communication, collaboration, mentorship, and peer learning mechanisms.}
    \item{Pandemic response was challenging but it was effective bringing together the community. Now that we are out of the pandemic, the momentum has more or less been lost. We do agree that for fulfilling pandemic needs, lots of things need to be developed during ``peace time'' - but how to motivate people during this time is the real challenge.}
\end{enumerate}
\subsection*{Notes}
\begin{itemize}
    \item{[Note any ongoing training pilots or institutional partners.]}
\end{itemize}
\subsection*{TODO}
\begin{itemize}
    \item{[Outline data or graphics needed to illustrate training pathways.]}
\end{itemize}

\section{Wrapping up}\label{sec7}
\subsection*{Key Points}
\begin{enumerate}
    \item{We need a community to facilitate conversations between different research groups for knowledge exchange and collaboration. This community would be a good infrastrucutre that can be used for communication and strategic alignment with policy makers.}
    \item{The activity of this community can help improving the modelling literacy in the policy and decision making communities.}
     \item{In epidemic response settings, such a community can also serve as a platform for reserve forces to be mobilised when needed.}
\end{enumerate}
\subsection*{Notes}
\begin{itemize}
    \item{[Capture closing narrative or calls-to-action that emerged in the workshop.]}
\end{itemize}
\subsection*{TODO}
\begin{itemize}
    \item{[List follow-up interviews or data points required for the conclusion.]}
\end{itemize}

\bibliographystyle{plainnat}
\bibliography{references}

\end{document}

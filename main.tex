\documentclass[10pt,a4paper]{article}
\usepackage[utf8]{inputenc}
\usepackage[T1]{fontenc}
\usepackage{geometry}
\geometry{a4paper, margin=2.5cm}
\usepackage{graphicx}
\usepackage{amsmath,amssymb,amsthm}
\usepackage{hyperref}
\usepackage{setspace}
\usepackage{titlesec}
\usepackage{natbib}
\usepackage{xcolor}
\usepackage{fancyhdr}

% Springer Nature style colors
\definecolor{SNBlue}{RGB}{0,102,153}

% Line spacing
\onehalfspacing

% Title formatting - unnumbered sections
\titleformat{\section}{\normalfont\Large\bfseries\color{SNBlue}}{}{0em}{}
\titleformat{\subsection}{\normalfont\large\bfseries}{}{0em}{}
\titleformat{\subsubsection}{\normalfont\normalsize\bfseries}{}{0em}{}

% Custom commands for Springer Nature style
\newcommand{\fnm}[1]{\textbf{#1}}
\newcommand{\sur}[1]{\textbf{#1}}
\newcommand{\email}[1]{\texttt{#1}}

% Author and affiliation formatting
\newcounter{affil}
\newcommand{\affil}[2][]{%
  \ifx\relax#1\relax
    \textsuperscript{#1}#2%
  \else
    \textsuperscript{#1}#2%
  \fi
}

% Abstract environment
\renewenvironment{abstract}{%
  \small
  \begin{center}%
    {\bfseries \abstractname\vspace{-.5em}\vspace{0pt}}%
  \end{center}%
  \quotation
}{%
  \endquotation
}

% Keywords command
\newcommand{\keywords}[1]{%
  \vspace{0.5cm}
  \noindent\textbf{Keywords:} #1
  \vspace{0.5cm}
}

% Remove page numbers for draft
\pagestyle{plain}
\pagenumbering{arabic}

\begin{document}

% Title
\begin{center}
    {\LARGE\bfseries\color{SNBlue} Towards a Community of Practice: Strengthening Infectious Disease Modelling Capacity in China}
\end{center}

\vspace{1cm}

% Authors - commented out for now
% \begin{center}
%     \textbf{[Author Name]\textsuperscript{1}$^{*}$}\quad\textbf{[Co-author Name]\textsuperscript{2}}
% \end{center}
%
% \vspace{0.3cm}
%
% % Affiliations
% \begin{center}
%     \textsuperscript{1}[Your Affiliation] \\
%     \textsuperscript{2}[Co-author Affiliation] \\
%     \vspace{0.2cm}
%     \textsuperscript{*}\textbf{Corresponding Author:} [Author Name] \\
%     Email: \texttt{[your.email@example.com]}
% \end{center}

\vspace{0.5cm}

\section{Overview}\label{sec1}
\subsection*{Key Points}

\begin{enumerate}
    \item{Learning from experiences during pandemic response across LMIC contexts.}
    \item{Modelling capacity for infectious disease decision making in LMICs remains fragmented compared with high-income settings, though targeted investments are growing.}
    \item{Briefly introduce the community workshop---its partners, participants, and aims as an LMIC-focused learning exchange.}
\end{enumerate}

\subsection*{Notes}
\begin{itemize}
    \item{[Add contextual notes or evidence references.]}
\end{itemize}

\paragraph{The COVID-19 pandemic renewed global recognition of the essential role of infectious disease modelling (IDM) in informing rapid, high-stakes public health decisions. However, persistent gaps in data sharing, analytical capacity and cross-jurisdictional collaboration remain. These gaps are most pronounced in resource-constrained settings in the Global South and show that effective policy use of infectious disease models depends not only on technical sophistication, but on the strength of the surrounding IDM ecosystem (cited by \cite{Leung2025}). Across low- and middle-income country (LMIC) settings, COVID-19 exposed how these weaknesses manifest in practice, including limited modelling and interpretive capacity, fragmented data systems and the lack of institutional mechanisms to embed modelling within routine decision-making, often resulting in delayed or under-utilised modelling outputs during periods of acute need (cited by \cite{Collins2024}).}
\paragraph{In Thailand, despite growing modelling expertise and increased use of IDM during COVID-19, siloed stakeholders and fragmented, low-accessibility datasets constrained model development and interpretation (cited by \cite{Sittimart2025}). China’s early pandemic response similarly demonstrated that technical capacity alone is insufficient: while models informed key epidemiological assessments and planning, their policy relevance was limited by data gaps and quality issues(cited by \cite{Feng2021}).
Importantly, these challenges are not confined to LMICs. UK-based assessments of the IDM community have identified comparable structural weaknesses, including fragmented coordination, unclear pathways for integrating modelling advice into policy, and unstable workforce and training pipelines, with limited progress over time. Together, these experiences indicate that weaknesses in IDM ecosystems represent a global systems challenge (cited by \cite{Sherratt2024, Sherratt2025}).}
\paragraph{Recognising these gaps and the need for more structured collaboration, the London School of Hygiene and Tropical Medicine (LSHTM), with funding from the Bill \& Melinda Gates Foundation (BMGF), convened the Advanced Technical Workshop on Infectious Disease Modelling (ATWIDM) at Duke Kunshan University from 26 to 28 May 2025. The workshop was designed as an learning exchange platform to strengthen modelling capacity, foster cross-institutional partnerships and explore future directions for IDM capacity building and ecosystem development in China.}
\paragraph{ATWIDM brought together 16 researchers specialising in infectious disease modelling, primarily from Chinese universities, academic institutes and the Chinese Center for Disease Control and Prevention (China CDC). Participants represented a broad range of experience, from doctoral students and postdoctoral fellows to senior academics and public health professionals. To broaden perspectives and enrich discussion, several international experts were invited to deliver keynote presentations and facilitate thematic sessions, contributing global insights to the workshop’s technical and strategic dialogues.}
\subsection*{TODO}
\begin{itemize}
    \item{Please refer to the two papers by Katherine Sherratt on assessing the IDM activity in the UK: \cite{Sherratt2024, Sherratt2025}.}
    \item{Mention the model use in policy lancet commission written by Mark et al.: \cite{Leung2025}.}
    \item{Mention the situation analysis of infectious disease modelling in Thailand: \cite{Sittimart2025}.}
    \item {Do we need a method section to outline the workshop structure and analysis methods?}
\end{itemize}

\section{Aligning Modelling Complexity With Real Decision Needs}\label{sec2}
\subsection*{Key Points}

\begin{enumerate}
    \item{There needs to be a balance between what is of research interest and what is of decision/ policy interest. For modellers specifically, this is a balance between complexity and operationality.}
    \begin{itemize}
    \item{Quote: Yang Liu-"with 16 age group, three broad gender groups, broad occupational groups, for many ethnic groups in the context of the uk that is about 99 million cells."}
    \item{Quote: Rachael-"I sometimes struggle between what is necessary and what is so good to have."}
    \item{Quote: Mark Jit-"I think the answer depends on who the decision maker is. If the decision maker is the central government, then they are making a single decision for the whole country, and what they are primarily interested in is the national average. In that case, using nationally aggregated estimates may be sufficient. However, if the decision maker is a provincial government, then locally derived data are much more relevant, and national averages may not adequately reflect local conditions."}
    \item{Quote: Joe Wu-"I think we need to, uh, strike a balance between how complex a model you need in order to address a problem and whether, uh, the models is complex enough to address the question."}
    \item {Quote: Kemin Zhu-"That is one of our major concerns that if we just simply add some extra complexity instead of making the model better, actually, it's a little bit over too complicated that we cannot, in practice, contact enough sensitivity analysis to test whether and to test it, which component is most important for the outcome results. "}
    \end{itemize}
    \item{In a decision/ policy support context, key research questions need to be identified together, between modellers and decision makers. This process needs to be iterative and collaborative.}
    \begin{itemize}
    \item{Quote: Kaja-"are those research questions like co-conceptualize with the decision makers of stakeholders? And how do you come up with these research questions?"
    "It's not just modellers alone, but you work with stakeholders in collaboration to define the research questions."}
    \item{Quote: Mark-"the best way is to actually bring the model as to interdisciplinary group that involves the domain experts, like immunologist, physicians, biologist to ask questions about the models, and then also the stakeholder there."}
    \end{itemize}
\end{enumerate}

\subsection*{Notes}
\begin{itemize}
  \item{Not all IDMs are intended to support decision-making. But those that do often are considered most impactful.}
  \item{Modellers sometimes chase after complexity, but complex models are not necessarily the best fit to support decision-making.}
  \item{How will decision makers understand what evidence they need?}  
  \item{Will decision makers use the evidence from a model if they do not necessarily understand it?}
  \item{Decision-informing models need to be dynamic - they need to be able to incorporate new data and knowledge as they become available.}
\end{itemize}

\subsection*{TODO}
\begin{itemize}
    \item{JK should go back to the database and see if there are some interesting quotes that you can pull to support these points.}
\end{itemize}

\section{Data}\label{sec3}
\subsection*{Key Points}
\begin{enumerate}
    \item{Data quality}
    \begin{itemize}
    \item{Quote: Cai Jun-"We need to make infectious disease data fair"}
    \item{Quote" Noriko-"We tried to collect the psychological data from the community. And at that time, the people who are handling the data collection in the field didn't collect age."}
    \end{itemize}
    
    \item{Data accessibility}
    \begin{itemize}
    \item{Quote: Noriko-"it's very difficult to get data, especially national data and we waited for 1 year to get to the national database which is a health insurance data in Japan."}
    \end{itemize}
    \item{Data representativeness}
    \begin{itemize}
    \item{Quote:Fang Hai"The first challenge is that the data often need to be disaggregated by mortality, complications, age, region, and year, because China is a very large and highly heterogeneous country, with substantial variation across provinces."
    "However, most of the available data come from specific local areas, and we often have to generalize these estimates to the national level, which may lead to underestimation of vaccine value and create challenges for setting policy priorities."}
    \end{itemize}
\end{enumerate}

\subsection*{Notes}
\begin{itemize}
    \item{[Summarise supporting evidence or workshop anecdotes.]}
\end{itemize}
\subsection*{TODO}
\begin{itemize}
    \item{[List analyses, figures, or interviews to include.]}
\end{itemize}

\section{Interdisciplinary Collaboration}\label{sec4}
\subsection*{Key Points}
\begin{enumerate}
    \item{IDM development is intrinsically interdisciplinary.}
    \begin{itemize}
    \item{Quote: Joe Wu-"When you have a group of people from ministry of health, you have clinicians, you have biologists, you have pathologists, uh, bioinformaticists, modelers sitting in the same room, they might have different definitions of infections.}
    \end{itemize}
    \item{This is where we can talk about the experience from the Canada experience - the language people are using are different.}
    \begin{itemize}
    \item{Work by the Pan-InfORM network in Canada showed that inconsistent definitions and misinterpretations of epidemiological terms could lead to divergent policy interpretations, prompting efforts to develop a common, jargon-free language. As part of this effort, Pan-InfORM conducted a review of common influenza modelling terms, which demonstrated differences, similarities, and discrepancies in terminology across studies. \cite{Seyed2015, Mehreen2021}.}
    \end{itemize}
    \item{We already discussed the problems and challenges with data - adding the interdisciplinary layer onto this makes it an even greater challenge.}
    \begin{itemize}
    \item{Quote:Yang Liu-"Example is when we were work different disciplines, working on the same project, a lot of time we have different ways to tell a story. So for a mathematicians, their interest may be making sure that there deriving equation in the most complete and clean way, whereas like public health and epidemiologist are interested in telling a story of making sure that individuals and readers from like broad health background could understand and use the information."}
    \item{Quote: Rachael-"when we work with people from different disciplines, their priorities might differ from public health".}
    \end{itemize}
\end{enumerate}
\subsection*{Notes}
\begin{itemize}
    \item{[Capture stakeholder perspectives or quotes to highlight tensions.]}
\end{itemize}
\subsection*{TODO}
\begin{itemize}
    \item{[Clarify which examples (e.g., Canada case) will be expanded.]}
\end{itemize}

% Jennifer will revisit the script for this section.
\section{AI in Modelling}\label{sec5}
\subsection*{Key Points}
\begin{enumerate}
    \item{AI is obviously a powerful tool for IDM development but it is not ready to replace humans yet. 
    It remains limited in producing transparent, reliable, and diversified models.}
    \begin{itemize}
    \item{Quote:Yang Liu-"ai is trying to give you one answer. I think relying on that model will lose the diversity."}
    \item{Quote:Kemin Zhu-"I always found them not that reliable, because they have the tendency to chase for what we call sota, which means a state of the art".}
    \item{Quote:Mark-"there's a lack of transparency because it's a purely statistical model."}
    \end{itemize}
    \item{Integrating AI into IDM developments requires computation power and performance tests.}
    \begin{itemize}
    \item{Quote:Kemin Zhu-"the first thing need for develop this capability is the most important thing is enough expert knowledge with a substantial training data of models."}
    \item{Quote:Yao Ye-"we must construct some kind of questions, some kind of test and we just like the ai to train to just at least pass this kind of test."}
    \end{itemize}
    \item{Ethical considerations. We need ``responsible use'' of AI in modelling - but what is being responsible mean?}
    \begin{itemize}
    \item{Quote:Mark-"I feel it's a bit like Plato's cave... you only see the shadows, but you never know whether they really represent the world behind."}
    \item{Quote:Shijia Ge-"If the model were developed by AI I think we need perhaps we need to develop a new indicator to tell it is right or not."}
    \end{itemize}
    \item{Working with AI now but also protects privacy and data security.}
    %TODO: @jinke what does it mean when they say they require "training tests"?
    \item{AI best practices as example? But who should be publishing these?}
\end{enumerate}
\subsection*{Notes}
\begin{itemize}
    \item{[Clarify open questions such as ``training tests'' requirement.]}
\end{itemize}
\subsection*{TODO}
\begin{itemize}
    \item{[Identify experts to interview on AI governance and infrastructure.]}
\end{itemize}

% Jennifer will revisit the script for this section.
\section{Building a Coherent and Continuous Training Ecosystem for IDM Capacity}\label{sec6}
\subsection*{Key Points}
\begin{enumerate}
    \item{AI-assisted learning may need to be carefully thought out. Good to have different standards of practice.}
    \begin{itemize}
    \item{Quote:Yang Liu-"I think it come cross my mind as like the role of AI in training as well…  
    quite a few people… copy and paste… into AI and then AI sort of saying, like really long answers…  
    how we need to work with the access to AI in the classroom setting… very conflicted… but it is a reality that we have to work with."}
    \end{itemize}
    \item{Capcity building and strengthening is segmented.}
    \begin{itemize}
    \item{Quote: Noriko-"When the course are introductory course, many courses are very great. But if you want to go to advanced, maybe not."}
    \item{Quote: Rachael-"What is lacking is hands on practice… moving from the course to the point of implementation… there's probably like an extra step that needs to happen."}
    \end{itemize}
    \item{Capacity building must go beyond technical skills to include leadership, communication, collaboration, mentorship, and peer learning mechanisms.}
    \begin{itemize}
    \item{Quote: Kemin Zhu-"one thing as a modelers and myself would like to have improve my capacity like leadership or organization or something like that, not just about model how do we get people together and communicate and collaborate?"}
    \end{itemize}
    \item{Pandemic response was challenging but it was effective bringing together the community. Now that we are out of the pandemic, the momentum has more or less been lost. We do agree that for fulfilling pandemic needs, lots of things need to be developed during ``peace time'' - but how to motivate people during this time is the real challenge.}
\end{enumerate}
\subsection*{Notes}
\begin{itemize}
    \item{[Note any ongoing training pilots or institutional partners.]}
    \item {[CMMID, China Field Epidemiology Training Program (CFETP),Nagasaki short course, gates foundation workshop]}
    \item {[China CDC task-model libray?-capacity building is highly demanding]}
\end{itemize}
\subsection*{TODO}
\begin{itemize}
    \item{[Outline data or graphics needed to illustrate training pathways.]}
\end{itemize}

\section{Wrapping up}\label{sec7}
\subsection*{Key Points}
\begin{enumerate}
    \item{We need a community to facilitate conversations between different research groups for knowledge exchange and collaboration. This community would be a good infrastrucutre that can be used for communication and strategic alignment with policy makers.}
    \begin{itemize}
    \item{Quote:Kemin Zhu-"If we have a community that will be entirely collaborated, I think it will help on improving - the impact of the research field on level of policy and decision making."}
    \item{Quote:Mark-"One is that's a community which actually addresses a lot of these people talk about like exchanges between different of groups, like different academic groups. Common resources, maybe we can take over organizing workshops like this sort of workshop recurring. Then the second thing is alignment with stakeholders, which is China CDC or the wider maybe government ministries, which actually, if there's a community, it makes it easier. But part of that is also improving modeling literacy so that the stakeholders see the value of modeling."}
    \item{Quote:Yan Niu-"in china, we can hold a such kind of committee modeling committee, infectious modeling committee."}
\end{itemize}        
    \item{The activity of this community can help improving the modelling literacy in the policy and decision making communities.}
     \item{In epidemic response settings, such a community can also serve as a platform for reserve forces to be mobilised when needed.}
\end{enumerate}
\subsection*{Notes}
\begin{itemize}
    \item{[Capture closing narrative or calls-to-action that emerged in the workshop.]}
\end{itemize}
\subsection*{TODO}
\begin{itemize}
    \item{[List follow-up interviews or data points required for the conclusion.]}
\end{itemize}

\bibliographystyle{plainnat}
\bibliography{references}

\end{document}
